\documentclass{article}
\usepackage[letterpaper, rmargin=3em, lmargin=3em, textheight=63em]{geometry}
\usepackage{fancyhdr}
\usepackage[spanish]{babel}
\usepackage[dvipsnames]{xcolor}
\usepackage{graphicx}
\usepackage{wrapfig}
\usepackage{setspace}
\usepackage{hyperref}

\usepackage{amssymb}
\usepackage{amsmath}

\usepackage{charter}
\usepackage{physics}

\usepackage{multicol}
\input{config.tex}

\begin{document}
\noindent \textbf{MA5701 Optimización no Lineal}\\
\textbf{Profesor:} Alejandro Jofré\\
\textbf{Auxiliar:} Benjamín Vera Vera


\begin{center}
	\Huge{\textbf{Auxiliar 10}}\\
	\textit{\large{Descenso de Gradiente Estocástico}}\\
	\normalsize
	4 de julio de 2025
\end{center}

\begin{enumerate}
	\item Sea \(f: \R^n \to \R\) fuertemente convexa de parámetro \(m\) y \(L\)-suave. Consideremos un algoritmo que consiste en búsquedas de línea exactas a lo largo de direcciones aleatorias. El esquema para obtener la siguiente iteración \(x^+\) desde \(x\) es el siguiente:
	      \begin{itemize}
		      \item Escoger \(v \sim \mathcal{N}(0, \sigma^2 I)\) independiente de las iteraciones previas.
		      \item Encontrar \(t_{\text{min}} = \argmin_{t} f(x + tv)\)
		      \item Definir \(x^+ = x + t_{\text{min}}v\)
	      \end{itemize}
	      El objetivo es probar que \(\E[ f(x^T) - f(x^*) ] \leq \epsilon\) siempre que
	      \[T \geq \frac{CnL}{m} \log\qty( \frac{f(x^0) - f(x^*)}{\epsilon})\]
	      para algún \(C>0\). Para ello, proceda como sigue:
	      \begin{enumerate}
		      \item Pruebe que, dado \(v \in \R^n, t > 0\) se tiene que
		            \[f(x + tv) \leq f(x) + \grad f^\top v + \frac{L}{2} t^2 \norm{v}^2.\]
		      \item Para \(v\) fijo, minimice sobre \(t\) ambos lados y obtenga una cota para \(f(x + t_{\text{min}}v)\).
		      \item Tomando esperanza a ambos lados sobre \(v\), utilice que \(\E[v_j^2 / \norm{v}^2] = \frac{1}{n}\) además de la cota conocida
		            \[\norm{\grad f(x)}^2 \geq 2m (f(x) - f(x^*))\]
		            para probar que
		            \[\E_v[f(x^+) - f(x^*)] \leq  \qty(1 - \frac{m}{nL}) \qty(f(x) - f(x^*)).\]
		      \item Concluya.
	      \end{enumerate}
\end{enumerate}

\end{document}
