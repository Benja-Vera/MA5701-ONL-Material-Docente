\documentclass{article}
\usepackage[letterpaper, rmargin=3em, lmargin=3em, textheight=63em]{geometry}
\usepackage{fancyhdr}
\usepackage[spanish]{babel}
\usepackage[dvipsnames]{xcolor}
\usepackage{graphicx}
\usepackage{wrapfig}
\usepackage{setspace}
\usepackage{hyperref}

\usepackage{amssymb}
\usepackage{amsmath}

\usepackage{charter}
\usepackage{physics}

\usepackage{multicol}
% GEOMETRY
\setlength{\parskip}{1em}
\pagestyle{fancy}
\lhead{Facultad de Ciencias Físicas y Matemáticas}
\rhead{Universidad de Chile}
\cfoot{ }

\renewcommand{\labelenumi}{\normalsize\bfseries P\arabic{enumi}.}
%\renewcommand{\labelenumii}{\normalsize\bfseries (\alph{enumii})}
\renewcommand{\labelenumiii}{\normalsize\bfseries \roman{enumiii})}

% Alfabeto
\newcommand{\A}{\mathcal{A}}
\newcommand{\B}{\mathcal{B}}
\newcommand{\C}{\mathcal{C}}
\newcommand{\E}{\mathcal{E}}
\newcommand{\F}{\mathcal{F}}
\newcommand{\I}{\mathcal{I}}
\newcommand{\K}{\mathcal{K}}
\renewcommand{\L}{\mathcal{L}}
\newcommand{\M}{\mathcal{M}}
\newcommand{\N}{\mathbb{N}}
\renewcommand{\P}{\mathcal{P}}
\newcommand{\Q}{\mathbb{Q}}
\newcommand{\R}{\mathbb{R}}
\renewcommand{\S}{\mathcal{S}}
\newcommand{\T}{\mathcal{T}}
\newcommand{\Z}{\mathbb{Z}}

\DeclareMathOperator{\sen}{sen}
\DeclareMathOperator{\senh}{senh}
\DeclareMathOperator{\tg}{tg}
\DeclareMathOperator{\dom}{dom}
\DeclareMathOperator{\dist}{dist}
\DeclareMathOperator{\argmin}{argmin}
\DeclareMathOperator{\Int}{int}

\renewcommand{\epsilon}{\varepsilon}
\renewcommand{\phi}{\varphi}
\newcommand{\dprod}[2]{\langle #1 , #2 \rangle}

\hypersetup{
    colorlinks=true,
    linkcolor=blue,
    filecolor=magenta,      
    urlcolor=blue,
}

\usepackage{amsthm}
\newtheorem{definition}{Definición}

\begin{document}
\noindent \textbf{MA5701 Optimización no Lineal}\\
\textbf{Profesor:} Alejandro Jofré\\
\textbf{Auxiliar:} Benjamín Vera Vera


\begin{center}
    \Huge{\textbf{Tarea Control}}\\
\textit{\large{Método de Penalización Externa}}\\
    \normalsize
	\today
\end{center}

En esta tarea estudiaremos una forma de resolver problemas convexos con restricciones lineales:
\begin{align*}
	\min_{x \in \R^n} & f(x) \\
	& Ax \leq b \\
	& Ex = e
\end{align*}
en que \(A \in \R^{m \times n}, b \in \R^m, E \in \R^{p \times n}, e \in \R^p\). La idea del método de penalización externa es cambiar a un problema irrestricto, cambiando la función objetivo y penalizando no estar en el conjunto factible del problema original, para esto se da la siguiente definición
\begin{definition}[Función de penalización]
	Una función continua \(\alpha: \R^n \to \R\) se dice \textit{función de penalización} si está dada por
	\[\alpha(x) = \sum_{i=1}^m \phi(A_{i\bullet} x - b_i) + \sum_{j=1}^p \psi(E_{j\bullet} - e_j)\]
	con \(\phi, \psi\) continuas en \(\R\) que cumplen:
	\begin{enumerate}
		\item[(i)] \(y \leq 0 \implies \phi(y) = 0, y > 0 \implies \phi(y) > 0\)
		\item[(ii)] \(z = 0 \implies \psi(z) = 0, z \neq 0 \implies \psi(z) > 0\)
	\end{enumerate}
\end{definition}
Considere para esta tarea las siguientes funciones de penalización comunmente utilizadas:
\begin{align*}
	\phi(y) &= \max(0, y)^2 \\
	\psi(z) &= z^2
\end{align*}
El \textit{algoritmo de penalización externa} viene entonces dado por
\begin{enumerate}
	\item[(0)] Sean \(k = 0, x_0 \in \R^n, \mu_0 > 0, \epsilon > 0, \beta > 1\).
	\item[(1)] Resolver el problema
	\begin{equation}\label{problem}
		\min_{x \in \R^n} \qty{f(x) + \mu_k \alpha(x)} \tag{$P_k$}
	\end{equation}
	sea \(x_{k+1}\) la solución de \ref{problem}.
	\item[(2)] \begin{itemize}
		\item Si \(\mu_k\alpha(x_{k+1}) < \epsilon\), parar.
		\item Si \(\mu_k\alpha(x_{k+1}) \geq \epsilon\), hacer \(\mu_{k+1} = \beta \mu_k, k \leftarrow k+1\) y volver a (1)
	\end{itemize}
\end{enumerate}
\begin{enumerate}
	\item Implemente los métodos de Nesterov y Gradiente Estocástico con paso constante a correr durante \(N=100\) iteraciones. La llamada del método debe ser del tipo
	\begin{itemize}
		\item \texttt{nesterov(f, gradf, x0, alpha, beta, N=100)}
		\item \texttt{sgd(f, gradf, x0, alpha, N=100)}
	\end{itemize}
	\item Considere los siguientes problemas
	\begin{align*}
		\min_{x,y}\; &x^2 + y^2 \\
		& x + y + 100 \leq 0
	\end{align*}
	\begin{align*}
		\min_{x,y}\; &(1 - x)^{3/2} + 100(y - x^2)^2 \\
		& x+y \leq 5 \\
		& x - 5y = 2
	\end{align*}
	Aplique el método de penalización para ambos con \(\mu_0 = 1, \beta = 2, \epsilon = 10^{-3}, (x_0, y_0) = (0, 0)\) utilizando el método de Nesterov para el paso (1). Evalúe en cuántas iteraciones para el método de penalización.
	\item Para \(n = 1000\), sea \(f: \R^n \to \R\) dada por
	\[f(x) = \sum_{i=1}^{n-1} \qty(50(x_{i+1} - x_i^2)^2 + (1 - x_i)^2)\]
	y considere el problema de minimizar \(f(x)\) sujeto a \(\sum_{i=1}^n x_i = n+1\). Aplique el algoritmo de penalización utilizando \texttt{sgd} para el paso (1). Evalúe cuántas iteraciones toma el método en parar.
\end{enumerate}

\end{document}
