\documentclass{article}
\usepackage[letterpaper, rmargin=3em, lmargin=3em, textheight=63em]{geometry}
\usepackage{fancyhdr}
\usepackage[spanish]{babel}
\usepackage[dvipsnames]{xcolor}
\usepackage{graphicx}
\usepackage{wrapfig}
\usepackage{setspace}
\usepackage{hyperref}

\usepackage{amssymb}
\usepackage{amsmath}

\usepackage{charter}
\usepackage{physics}

\usepackage{multicol}
\input{config.tex}

\begin{document}
\noindent \textbf{MA5701 Optimización no Lineal}\\
\textbf{Profesor:} Alejandro Jofré\\
\textbf{Auxiliar:} Benjamín Vera Vera


\begin{center}
    \Huge{\textbf{Auxiliar 4}}\\
	\textit{\large{Ritmos de convergencia}}\\
    \normalsize
	11 de abril de 2025
\end{center}

\begin{enumerate}
	\item \textbf{(El método de Newton)} Considere \( f \in C^2 \) con Hessiano \( \nabla^2 f(x) \) \( L \)-Lipschitz en una vecindad de \( x^* = \argmin(f) \) en el que se cumplen las condiciones suficientes de optimalidad de segundo orden. Con esto, considere la iteración a partir de \( x^0 \) dada por
\[
x^{k+1} = x^k - \nabla^2 f(x^k)^{-1} \nabla f(x^k)
\]
\begin{enumerate}
    \item Pruebe que existe una vecindad \( V \) de \( x^* \) tal que si \( x^0 \in V \), entonces \( x^k \to x^* \) cuadráticamente.
    \item Pruebe que en esta vecindad, la secuencia de las normas \( \left\{ \norm{\nabla f(x^k)} \right\}_{k \in \mathbb{N}} \) converge a \( 0 \) cuadráticamente.
\end{enumerate}
\end{enumerate}

\end{document}
