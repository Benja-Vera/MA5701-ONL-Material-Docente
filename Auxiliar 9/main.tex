\documentclass{article}
\usepackage[letterpaper, rmargin=3em, lmargin=3em, textheight=63em]{geometry}
\usepackage{fancyhdr}
\usepackage[spanish]{babel}
\usepackage[dvipsnames]{xcolor}
\usepackage{graphicx}
\usepackage{wrapfig}
\usepackage{setspace}
\usepackage{hyperref}

\usepackage{amssymb}
\usepackage{amsmath}

\usepackage{charter}
\usepackage{physics}

\usepackage{multicol}
% GEOMETRY
\setlength{\parskip}{1em}
\pagestyle{fancy}
\lhead{Facultad de Ciencias Físicas y Matemáticas}
\rhead{Universidad de Chile}
\cfoot{ }

\renewcommand{\labelenumi}{\normalsize\bfseries P\arabic{enumi}.}
%\renewcommand{\labelenumii}{\normalsize\bfseries (\alph{enumii})}
\renewcommand{\labelenumiii}{\normalsize\bfseries \roman{enumiii})}

% Alfabeto
\newcommand{\A}{\mathcal{A}}
\newcommand{\B}{\mathcal{B}}
\newcommand{\C}{\mathcal{C}}
\newcommand{\E}{\mathcal{E}}
\newcommand{\F}{\mathcal{F}}
\newcommand{\I}{\mathcal{I}}
\newcommand{\K}{\mathcal{K}}
\renewcommand{\L}{\mathcal{L}}
\newcommand{\M}{\mathcal{M}}
\newcommand{\N}{\mathbb{N}}
\renewcommand{\P}{\mathcal{P}}
\newcommand{\Q}{\mathbb{Q}}
\newcommand{\R}{\mathbb{R}}
\renewcommand{\S}{\mathcal{S}}
\newcommand{\T}{\mathcal{T}}
\newcommand{\Z}{\mathbb{Z}}

\DeclareMathOperator{\sen}{sen}
\DeclareMathOperator{\senh}{senh}
\DeclareMathOperator{\tg}{tg}
\DeclareMathOperator{\dom}{dom}
\DeclareMathOperator{\dist}{dist}
\DeclareMathOperator{\argmin}{argmin}
\DeclareMathOperator{\Int}{int}

\renewcommand{\epsilon}{\varepsilon}
\renewcommand{\phi}{\varphi}
\newcommand{\dprod}[2]{\langle #1 , #2 \rangle}

\hypersetup{
    colorlinks=true,
    linkcolor=blue,
    filecolor=magenta,      
    urlcolor=blue,
}

\begin{document}
\noindent \textbf{MA5701 Optimización no Lineal}\\
\textbf{Profesor:} Alejandro Jofré\\
\textbf{Auxiliar:} Benjamín Vera Vera


\begin{center}
	\Huge{\textbf{Auxiliar 9}}\\
	\textit{\large{Preparación C2}}\\
	\normalsize
	13 de junio de 2025
\end{center}

\begin{enumerate}
	\item Sea \(f\) convexa y considere el siguiente esquema partiendo de \(x_0 \in \R^n, y_0 = x_0\):
	\begin{equation} \label{nesterov}
		\left\{\begin{aligned}
			x_k &= y_{k-1} - s \grad f (y_{k-1}) \\
			y_k &= x_k + \frac{k-3}{k} (x_k - x_{k-1}).
		\end{aligned}\right.
	\end{equation}
	Deduzca a partir de una aproximación de Taylor la siguiente ecuación diferencial para una trayectoria que sigue \(\qty{x_k}\):
	\begin{align} \label{ODE}
		\ddot{X} + \frac{3}{t} \dot{X} + \grad f(X) &= 0 \\
		X(0) &= x_0 \\
		\dot{X}(0) &= 0
	\end{align}

	\item Se puede probar que \ref{ODE} admite una única solución global para \(t \geq 0\) y que el esquema \ref{nesterov} converge a ella en el siguiente sentido:
	\[\lim_{s \to 0} \max_{0 \leq k \leq T/\sqrt{s}} \norm{x_k - X(\sqrt{s}k)} = 0, \qquad \forall T \geq 0\]
	\begin{enumerate}
		\item Suponga además que \( \ddot{X}(0) := \lim_{t \to 0} \ddot{X}(t)\) existe. Pruebe la siguiente expansión asintótica para \(t\) cercano a 0:
		\[X(t) = - \frac{\grad f(x_0)}{8}t^2 + x_0 + o(t^2)\]
		\textit{Indicación:} Utilice el teorema del valor medio.
		\item Sea \(f\) convexa con \(\grad f\) lipschitz y sea \(X(t)\) la única solución del problema \ref{ODE}. Pruebe que para \(t > 0\):
		\[f(X(t)) - f^* \leq \frac{2 \norm{x_0 - x^*}^2}{t^2}\]
	\end{enumerate}
\end{enumerate}

\end{document}
